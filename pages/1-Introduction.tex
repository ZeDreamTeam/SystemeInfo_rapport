\section{Présentation du projet}

    \subsection*{Réalisation}
    Ce rapport à pour but de montrer et discuter des points importants que nous avons pu rencontrer tout au long du projet \textbf{Système Informatique}, en 4ème année à l'INSA.  Nous avons donc, sur un semestre complet, pu suivre des TPs comprenant 2 parties principales : la réalisation d'un \textbf{Compilateur}, à l'aide de lex et de yacc, ainsi que celle d'un \textbf{microprocesseur} à l'aide notamment de VHDL.
    
    \subsection*{Organisation}
    Notre organisation durant ces TPs a été la suivante. Nous avons commencer par travailler à deux sur le compilateur. Nous nous sommes séparés le travail, de manière à ce que chacun puisse ajouter sa pierre à l'édifice. Pour faciliter les choses, nous avons aussi décider de travailler en utilisant un gestionnaire de version \textbf{Git}.
    
    Après avoir réalisé un compilateur fonctionnel, nous sommes passé à l'étape suivante en s'attaquant à notre microprocesseur. Nous avons commencé par se séparer le travail en réalisant les divers composants nécessaires, avant de s'attaquer à la création de pipelines, et donc à la mise en relation entre ces différents composants. 
    
    Arrivé à ce niveau là, nous avons décidé de nous séparer les tâches : l'un d'entre nous s'est attelé à la tâche de terminer le compilateur, en lui rajoutant des fonctions supplémentaires, tandis que l'autre s'est mit à celle de terminer l'implémentation du microprocesseur.
    
    Nous pensons que cette séparation des tâches nous a permis d'avancer le projet de façon correct, et d'arriver à un point satisfaisant quant à notre réalisation.
    
    \subsection*{Liens}
    Vous pouvez retrouver l'ensemble de nos travaux sur Github.
    
    Le \textbf{compilateur} se trouve sur le repository suivant \url{https://github.com/ZeDreamTeam/Compilateur/}
    
    Le \textbf{microprocesseur} se trouve à l'adresse suivante : \url{https://github.com/ZeDreamTeam/MicroProcesseur}