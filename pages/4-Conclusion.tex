\section{Conclusion}
    Ce projet maintenant achevé, nous aura donc permis l'implémentation d'un compilateur de language C plutôt avancé, et d'un microprocesseur permettant d'exécuter un assembleur personnalisé.
    
    Nous sommes assez fier de la réalisation du processeur comme du compilateur. Les deux nous ont permis d'approfondir nos connaissances en Automate et Langage et en VHDL, mais aussi d'en appliquer d'autre, comme du C. Nous pensons aujourd'hui avoir une vision plus clair de ce qu'il se passe lors d'un processus de compilation simplifié, et de l'évolution de l'environnement d'éxecution d'un programme.
    
    Les travaux sur ce projet nous ont paru important pour notre formation, puisque non seulement nous permet d'acquérir un certains nombres de nouveaux savoir et savoir faire, mais en plus nous a semblé ludique et nous a facilement interessé. De plus, notre chargé de TP (Mr. Benoit Morgan) a été très instructif et nous a aider à mieux comprendre le projet et certaines parties, difficile a assimilés, de ce dernier.
    
    Nous regrettons cependant de ne pas avoir pu relier les deux parties pour éxecuter notre assembleur sur notre microprocesseur. La création d'un interpréteur nous a paru intéressante, mais pas suffisante pour montrer les capacités de notre code généré. Mais nous savons bien qu'aller plus loin en éxecutant notre code sur notre microprocesseur aurait été difficile, et aurait demandé un temps supplémentaire considérable. En effet, notre assembleur étant orienté mémoire, et notre processeur orienté registre, il aurait fallu réaliser un cross assembler, adapté à notre language et permettant de passer de l'un à l'autre facilement !